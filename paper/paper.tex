% LeanYjs paper scaffold
\documentclass[11pt]{article}

% Encoding and language basics (kept simple for portability)
\usepackage[T1]{fontenc}
\usepackage[utf8]{inputenc}
\usepackage{lmodern}

% Math and symbols
\usepackage{amsmath,amssymb,amsthm,mathtools}

% Layout and graphics
\usepackage[margin=1in]{geometry}
\usepackage{graphicx}
\usepackage{xcolor}
\usepackage{microtype}

% Lists and tables
\usepackage{enumitem}
\usepackage{booktabs}

% Hyperlinks and clever refs
\usepackage{hyperref}
\hypersetup{
  colorlinks=true,
  linkcolor=blue,
  citecolor=blue,
  urlcolor=blue,
  pdftitle={Lean-Yjs: Placeholder Title},
  pdfauthor={Your Name}
}
\usepackage[nameinlink,capitalize]{cleveref}

% Citations via natbib + BibTeX (arXiv-friendly)
\usepackage[square,sort,comma,numbers]{natbib}

% Code listings (no external deps like minted/pygmentize)
\usepackage{listings}
\lstset{
  basicstyle=\ttfamily\small,
  numbers=left,
  numberstyle=\scriptsize, 
  numbersep=8pt,
  showstringspaces=false,
  keywordstyle=\color{blue!70!black},
  commentstyle=\color{green!40!black},
  stringstyle=\color{orange!60!black},
  frame=single,
  framerule=0.3pt,
  breaklines=true
}

% Theorem environments
\theoremstyle{plain}
\newtheorem{theorem}{Theorem}
\newtheorem{lemma}[theorem]{Lemma}
\theoremstyle{definition}
\newtheorem{definition}[theorem]{Definition}
\theoremstyle{remark}
\newtheorem{remark}[theorem]{Remark}

% User macros
\newcommand{\Lean}{\textsf{Lean}}  % example macro
\newcommand{\Yjs}{\textsf{Yjs}}

\title{Lean-Yjs: Formalizing Yjs Item Order in Lean}
\author{Your Name}
\date{\today}

\begin{document}

\maketitle

\begin{abstract}
This is a minimal scaffold for a paper accompanying the Lean-Yjs project. Replace this abstract with your own.
\end{abstract}

\section{Introduction}
Briefly introduce collaborative editing CRDTs, the \Yjs{} data model, and the motivation for formal verification in \Lean{}.

\paragraph{Contributions.} List key contributions, e.g., a formal item order, totality/antisymmetry/transitivity proofs, and an integration invariant.

\section{Background}
Summarize the YATA lineage and how \Yjs{} differs (e.g., right-origin tie-breaking). Cite prior work \citep{yata}.

\section{Item Order}
Define the item order used by \Yjs{}, its edge-cases, and how it deviates from simplified presentations.

\begin{definition}[Item order]
Provide a crisp, math-level definition consistent with the mechanized version.
\end{definition}

\begin{theorem}[Totality]
State and sketch proof ideas. Refer to mechanized proof scripts.
\end{theorem}

\section{Integration Invariant}
Describe the loop invariant guaranteeing that inserts preserve structure.

\section{Evaluation}
Optional: discuss scale of proofs, proof engineering choices, and limitations.

\section{Related Work}
Place the result among CRDT verification efforts and editor concurrency.

\section{Conclusion}
Summarize findings and potential future directions.

\paragraph{Artifacts.} Repository: \url{https://github.com/iasakura/lean-yjs}.

\bibliographystyle{unsrtnat}
\bibliography{refs}

\end{document}

